\documentclass[12pt]{article}
\usepackage{listings}
\usepackage{graphicx}
\usepackage{subfig}
\usepackage{color}
\definecolor{dkgreen}{rgb}{0,0.6,0}
\definecolor{gray}{rgb}{0.5,0.5,0.5}
\definecolor{mauve}{rgb}{0.58,0,0.82}
\lstset{frame=tb,
  language=Python,
  aboveskip=3mm,
  belowskip=3mm,
  showstringspaces=false,
  columns=flexible,
  basicstyle={\small\ttfamily},
  numbers=none,
  numberstyle=\tiny\color{gray},
  keywordstyle=\color{blue},
  commentstyle=\color{dkgreen},
  stringstyle=\color{mauve},
  breaklines=true,
  breakatwhitespace=true,
  tabsize=3
}\title{Notes on performance of various networks on bee problem}
\author{
	Tommy 
}
\date{\today}


\begin{document}
\maketitle

\section{Introduction}

\paragraph{Not ensuring class balance.}
Now there are no masked ``no data'' pixels - the image is fully segmented into bee/not bee.
This means that pixel-wise class balance is nonexistent.
\includegraphics[width=0.5\textwidth]{example_images/class_mask.png}
The output of the neural network is all not bee.

\paragraph{Output probabilities, and negative training example size.}
The larger boxes result in the network not outputting a p > 0.5 for any pixel in the bee category.
Using a negative size of 1 results in probabilities in the bee category of 90$\%$. 
To examine this issue, I trained different
networks using different box sizes (6 of them). The results are in the figure below. The larger the
box size, the higher probability that the negative points contain a bee. This could be dealt with 
by segmenting not-bee parts of an image, or choosing not-bee samples more intelligently.

\begin{figure}
\centering
\subfloat[first subfigure]{\label{sfig:a}\includegraphics[scale=0.5]{example_images/preds_vs_ground_truth_box0.png}}\hfill
\subfloat[second subfigure]{\label{sfig:b}\includegraphics[scale=0.5]{example_images/preds_vs_ground_truth_box3.png}}\hfill
\caption{Preliminary results.}
\label{fig:test}
\end{figure}

I'm pretty sure the bad results with a large box size are because the random boxes are overlapping
bees in some cases. All of the results above were obtained with a simple architecture:
\begin{lstlisting}
    model = tf.keras.Sequential()
    model.add(tf.keras.layers.Conv2D(filters=32, kernel_size=8, padding='same', activation='relu',
        input_shape=image_shape, data_format='channels_last'))
    model.add(tf.keras.layers.Conv2D(filters=64, kernel_size=4, padding='same', activation='relu'))
    model.add(tf.keras.layers.Conv2D(filters=32, kernel_size=4, padding='same', activation='relu'))
    model.add(tf.keras.layers.Conv2D(filters=16, kernel_size=2, padding='same', activation='relu'))
    model.add(tf.keras.layers.Dropout(0.5))
    model.add(tf.keras.layers.Conv2D(filters=n_classes, kernel_size=2, padding='same',
        activation='softmax'))
\end{lstlisting}
The results from generating negative samples where the is no bee in the negative sample are below.

\begin{figure}
\centering
\subfloat[first
subfigure]{\label{sfig:a}\includegraphics[width=.18\textwidth,height=2cm]{example_images/preds_vs_ground_truth_smart_box0.png}}\hfill
\subfloat[second
subfigure]{\label{sfig:b}\includegraphics[width=.18\textwidth,height=2cm]{example_images/preds_vs_ground_truth_smart_box3.png}}\hfill
\subfloat[third
subfigure]{\label{sfig:c}\includegraphics[width=.18\textwidth,height=2cm]{example_images/preds_vs_ground_truth_smart_box6.png}}\hfill
\subfloat[fourth
subfigure]{\label{sfig:d}\includegraphics[width=.18\textwidth,height=2cm]{example_images/preds_vs_ground_truth_smart_box9.png}}\hfill
\subfloat[fifth
subfigure]{\label{sfig:e}\includegraphics[width=.18\textwidth,height=2cm]{example_images/preds_vs_ground_truth_smart_box12.png}}\\
\subfloat[sixth
subfigure]{\label{sfig:f}\includegraphics[width=.18\textwidth,height=2cm]{example_images/preds_vs_ground_truth_smart_box15.png}}\hfill
\subfloat[seventh
subfigure]{\label{sfig:g}\includegraphics[width=.18\textwidth,height=2cm]{example_images/preds_vs_ground_truth_smart_box18.png}}\hfill
\subfloat[eighth
subfigure]{\label{sfig:h}\includegraphics[width=.18\textwidth,height=2cm]{example_images/preds_vs_ground_truth_smart_box20.png}}\hfill
\subfloat[ninth
subfigure]{\label{sfig:i}\includegraphics[width=.18\textwidth,height=2cm]{example_images/preds_vs_ground_truth_smart_box23.png}}\hfill
\subfloat[tenth
subfigure]{\label{sfig:j}\includegraphics[width=.18\textwidth,height=2cm]{example_images/preds_vs_ground_truth_smart_box29.png}}\\
\caption{Preliminary results.}
\label{fig:test}
\end{figure}


\section{Model experimentation}






\end{document}


